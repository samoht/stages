\documentclass{standalone}

\input{caml.tex}

\usepackage{tikz}
\usetikzlibrary[arrows]

\begin{document}
\begin{tikzpicture}[align=center,node distance=4em]

%\draw[step=1cm] (-10,-10) grid (10, 10);

\tikzstyle{txt}=[
	text width=2em,
	color=darkalu
]
\tikzstyle{index}=[
	text width=1em,
	text height=1em,
	circle,
	draw=scarletred!50,
	fill=scarletred!20
]
\tikzstyle{node}=[
	text width=1em,
	text height=1em,
	circle,
	draw=orange!50,
	fill=orange!20
]
\tikzstyle{elt}=[
	text width=1.5em,
	text height=3em,
	anchor=north,
	rectangle,
	draw=skyblue!50,
	fill= skyblue!20
]
\tikzstyle{lbox}=[
	text width=0.4cm,
	text height=0.2em,
	rectangle,
	anchor=east
]
\tikzstyle{ltxt}=[
	text height=0.4em,
	rectangle,
	anchor=west
]
\tikzstyle{arr}=[->, >=stealth, shorten <=1pt, shorten >=1pt]
\tikzstyle{arrelt}=[arr, dashed]

\draw[fill=chocolate!10, draw=white]
	(-3, 3.7) -- (0, 0.7) -- (0, -0.7) -- (-3.7, 3) -- cycle;
\draw[fill=butter!10, draw=white]
	(4, 4.7) -- (0, 0.7) -- (0, -0.7) -- (4.7, 4) -- cycle;

\draw[fill=chocolate!10, draw=white]
	(4, 1.7) -- (6, -0.3) -- (6, -1.7) -- (3.3, 1) -- cycle;
\draw[fill=butter!10, draw=white]
	(7, 0.7) -- (6, -0.3) -- (6, -1.7) -- (7.7, 0) -- cycle;

\coordinate (root) at (0, 6);
\node[index] (index) at (0, 4.5) {};
\draw[arr, thick] (root) -> (index);
\node[txt] at (0, 4.5) {\scriptsize\ttfamily I0};

\node[node] (n01) at (-3, 3) {};
\node[txt] at (-3, 3) {\scriptsize\ttfamily n01};
\node[node] (n02) at (-2, 2) {};
\node[txt] at (-2, 2) {\scriptsize\ttfamily n02};
\node[node] (n03) at (-1, 1) {};
\node[txt] at (-1, 1) {\scriptsize\ttfamily n03};
\node[node] (n04) at (0, 0) {};
\node[txt] at (0, 0) {\scriptsize\ttfamily n04};
\node[node] (n05) at (1, 1) {};
\node[txt] at (1, 1) {\scriptsize\ttfamily n05};
\node[node] (n06) at (2, 2) {};
\node[txt] at (2, 2) {\scriptsize\ttfamily n06};
\node[node] (n07) at (4, 4) {};
\node[txt] at (4, 4) {\scriptsize\ttfamily n07};

\node[index] (branch) at (3, 3) {};
\node[txt] at (3, 3) {\scriptsize\ttfamily I1};

\node[node] (n11) at (4, 1) {};
\node[txt] at (4, 1) {\scriptsize\ttfamily n11};
\node[node] (n12) at (5, 0) {};
\node[txt] at (5, 0) {\scriptsize\ttfamily n12};
\node[node] (n13) at (6, -1) {};
\node[txt] at (6, -1) {\scriptsize\ttfamily n13};
\node[node] (n14) at (7, 0) {};
\node[txt] at (7, 0) {\scriptsize\ttfamily n14};

\draw[arr] (index) -> (n01) node[sloped, midway, above]
	{\footnotesize\ttfamily top};
\draw[arr] (index) -> (n07) node[sloped, midway, above]
	{\footnotesize\ttfamily bottom};

\draw[arr] (branch) -> (n11) node[sloped, midway, above]
	{\footnotesize\ttfamily top};
\draw[arr] (branch) -> (n14) node[sloped, midway, above]
	{\footnotesize\ttfamily bottom};

\draw[arr] (n01) -> (n02);
\draw[arr] (n02) -> (n03);
\draw[arr] (n03) -> (n04);

\draw[arr] (n07) -> (branch);
\draw[arr] (branch) -> (n06);
\draw[arr] (n06) -> (n05);
\draw[arr] (n05) -> (n04);

\draw[arr] (n11) -> (n12);
\draw[arr] (n12) -> (n13);
\draw[arr] (n14) -> (n13);

\node[elt] (elt0) at (-3, -2) {};
\node[elt] (elt1) at (-1.5, -2) {};
\node[elt] (elt2) at (0, -2) {};
\node[elt] (elt3) at (1.5, -2) {};
\node[elt] (elt4) at (3, -2) {};
\node[elt] (elt5) at (4.5, -2) {};
\node[elt] (elt6) at (6, -2) {};
\node[elt] (elt7) at (7.5, -2) {};

\draw[arrelt] (n01) -> (elt0.north);
\draw[arrelt] (n02) -> (elt1.94);
\draw[arrelt] (n03) -> (elt1.86);
\draw[arrelt] (n04) -> (elt2.north);
\draw[arrelt] (n05) -> (elt3.94);
\draw[arrelt] (n06) -> (elt4.north);
\draw[arrelt] (n07) -> (elt3.86);
\draw[arrelt] (n11) -> (elt5.94);
\draw[arrelt] (n12) -> (elt5.86);
\draw[arrelt] (n13) -> (elt6.north);
\draw[arrelt] (n14) -> (elt7.north);

\node[lbox, draw=scarletred!50, fill=scarletred!20] at (6, 5) {};
\node[ltxt] at (6, 5) {\footnotesize\ttfamily Index};
\node[lbox, draw=orange!50, fill=orange!20] at (6, 4.5) {};
\node[ltxt] at (6, 4.5) {\footnotesize\ttfamily Node};
\node[lbox, draw=skyblue!50, fill=skyblue!20] at (6, 4) {};
\node[ltxt] at (6, 4) {\footnotesize\ttfamily Elt};
\node[lbox, draw=white, fill=chocolate!10] at (6, 3.5) {};
\node[ltxt] at (6, 3.5) {\footnotesize\ttfamily pop list};
\node[lbox, draw=white, fill=butter!10] at (6, 3) {};
\node[ltxt] at (6, 3) {\footnotesize\ttfamily push list};

\end{tikzpicture}
\end{document}