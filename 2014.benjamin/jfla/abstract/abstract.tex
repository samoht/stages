\documentclass{article}

\usepackage[utf8]{inputenc}
\usepackage[T1]{fontenc}
\usepackage[english]{babel}
\usepackage{charter}

\usepackage{xspace}
\newcommand{\git}{Git\xspace}
\newcommand{\irmin}{Irmin\xspace}
\newcommand{\mirage}{MirageOS}

\begin{document}

\begin{abstract}
%Nowadays, typical applications spawn a large set of heterogeneous computing and storage devices with very different technical properties.
%Distributed applications running on these devices need to be able to choose between different kinds of policies:
%Where to store what ? What level of trust? When to migrate data ?
%Irmin is a library to persist and synchronize distributed data structures.
%It is, as are all other components of Mirage OS, a collection of libraries designed to solve different flavours of the challenges raised by the CAP theorem.

\irmin, the new \git-like storage layer of \mirage, is library to persist and synchronize distributed data structures.
It is, as are all other components of \mirage, a collection of libraries designed to solve different flavours of challenges.
In the case of \irmin, this challenges are those raised by the CAP theorem.
The solution provided by \irmin involve merging data structures.

In this paper, we (I?) present two first mergeable data structure -- the queue and the rope data structures --, directly built on the top of the internal \irmin store.
After explaining the algorithm behind both of them, I will analyse these data structure in terms of correctness and efficiency.

\end{abstract}

\end{document}